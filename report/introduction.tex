\documentclass[report.tex]{subfiles}

\begin{document}

    \section*{\centering Introduction}

    Reinforcement is another large area in machine learning alongside supervised and unsupervised learning. The idea behind reinforcement learning is learning by interacting with an environment, which is how learning in nature happens very often. A child putting its hands on a hot stove will quickly feel a painful sensation and remember that was a terrible idea, and ultimately learn from it. Learning can also have positive effects such as when a mouse reaches the end of the labyrinth and finds a piece of cheese.

    Reinforcement is then a computational approach to this kind of learning by having an so called \textit{agent} interact with an environment and learn how it responds to different actions. The goal is to create learning methods that can learn an optimal \textit{policy}, a mapping from the current environmental state to a action decision, according to a behavior we want to achieve. That might be from learning the computer to play tic-tac-toe to more complex behavior such as flying a helicopter where the agent do not posses full control of the environment.



    \subsection*{Questions}

    \begin{enumerate}
        \item Is the state representation important for learning?
        \item Is it important for the reward function to reinforce behavior we want the agent to learn?
        \item Is punishing bad behavior equivalent to reinforcing good behavior?
        \item Is it better if the reward function combines reinforcement and punishment?
    \end{enumerate}

\end{document}
